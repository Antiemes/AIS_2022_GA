\documentclass[conference]{IEEEtran}
\IEEEoverridecommandlockouts
% The preceding line is only needed to identify funding in the first footnote. If that is unneeded, please comment it out.
\usepackage{cite}
\usepackage{amsmath,amssymb,amsfonts}
\usepackage{algorithmic}
\usepackage{graphicx}
\usepackage{textcomp}
\usepackage{xcolor}
%\usepackage{booktabs}
\usepackage{tabularray}
\UseTblrLibrary{booktabs}
\def\BibTeX{{\rm B\kern-.05em{\sc i\kern-.025em b}\kern-.08em
    T\kern-.1667em\lower.7ex\hbox{E}\kern-.125emX}}
\begin{document}

\title{Genetic algorithm based image approximation with geometrical shapes}

\author{\IEEEauthorblockN{Gergely Vakulya}
\IEEEauthorblockA{\small \textit{Alba Regia Technical Faculty} \\
\textit{Óbuda University}\\
\textit{Székesfehérvár, Hungary}\\
\textit{vakulya.gergely@amk.uni-obuda.hu}
}}

\maketitle

\begin{abstract}
In this paper the design parameters of a
genetic algorithm (GA) based image
approximation / compression algorithm are
analyzed. The efficiency using different
approximation shapes and with different GA
parameters (population size, mutation rate)
are evaluated using test runs. Recommended
parameter sets for different scenarios are
provided.
\end{abstract}

\begin{IEEEkeywords}
genetic algorithm, image approximation
\end{IEEEkeywords}

\section{Introduction}

The genetic algorithm \cite{ga-book} concept can be efficiently used in many
optimization problems to provide an alternative solution. It can be used is
several areas from data mining \cite{barman2017} to neural networks and deep learning \cite{such2017}.
GA does not guarantee the optimum, and the quality
of the approximation highly depends on the implementation and on the carefully
chosen design parameters.

One large area, where GA can efficiently be used, is image processing \cite{ga-imgseg},
e.g. image approximation.

In \cite{vakulya2021} an image approximation algorithm was presented using
arbitrary triangles to approximate the back and white areas of images.
Using different shapes, however, can lead to different
results. In this paper additional shapes (rectangles and circles) will be used
for the approximation, among the triangles.

The efficiency of the algorithm is highly affected by the design parameters. In
this paper the effect of the used shapes, the maximum number of shapes, the
mutation and crossover ratio and the distribution and the parameters of the
random numbers used in generation of the new shapes and during the mutations
are evaluated.

\section{Related work}

Genetic algorithms \cite{ga-book} are widely used \cite{such2017}, primarily for solving problems, where
the number of parameters is  too high and no quick solution is known. For
most of those problems finding the optimal solution is not critical, and
an approximation (or maybe any possible solution) can be acceptable.

An approximate solution can be given to classic NP-complete problems,
Vehicle Routing Problem (VRP) \cite{ga-vrp} and Travelling Salesman Problem (TSP) \cite{ga-tsp},
with genetic algorithms. Another two typical applications are transporting \cite{ga-transport}
and scheduling \cite{ga-scheduling}.

In mechanical engineering genetic algorithms can be used to optimize mechanical
components, e.g. to make support elements with minimum weight (and from
minimum amount of material) and with maximal strength \cite{bhoskar2015}.

Different areas, like fuzzy logic \cite{buckley1994}, data mining \cite{barman2017} or
scheduling \cite{gonccalves2005} can efficiently combined with genetic algorithms.

Genetic algorithms can be effectively used for economic applications \cite{dawid1998},
e.g. for automatized trading \cite{ga-trading} or portfolio analysis \cite{ga-portfolio}.

A promising application is using genetic algorithm to optimize the
weights of a neural network instead of using back propagation \cite{hecht1992, such2017}.
Two interesting applications of neuro-evolution are an automated player
for the classic Nintendo Super Mario Brothers video game (MarIO) \cite{mar-io}
and an application, where a virtual robot battle player is controlled by a
neuro-evolutionary algorithm \cite{ga-robocode}.

A good application of genetic algorithms is image processing \cite{image-proc}, e.g.
image segmentation \cite{ga-imgseg} or image enhancement \cite{ga-imgenh}. This paper
will focus on a similar application: image approximation \cite{vakulya2021}.

\section{The GA-based image compression}

\subsection{Problem statement}


\subsection{Chromosome representation}

The required amount of data to represent the size and the position of the shapes
is different in each case. In case of triangles the coordinates of the three
vertices are required to store, which means a total of 6 numerical parameters.
A circle can be defined with the center point and the radius, summing up to 3 parameters.
Rectangles can be given with the two opposite vertices, resulting in 4 parameters.
Finally, the shapes can be mixed, as well.

\subsection{Fitness function}


\subsection{Genetic operators}


\subsection{Program flow}


\subsection{Software architecture}


\section{Test results}

Figures \ref{sherlock-6} and \ref{cat-6}

\begin{figure}[htbp]
	\centering
	\includegraphics[width=0.35\textwidth]{fig/sherlock6.png}
	\caption{The best matching Sherlock image after $g$ generations.}
	\label{sherlock-6}
\end{figure}

\begin{figure}[htbp]
	\centering
	\includegraphics[width=0.35\textwidth]{fig/cat6.png}
	\caption{The best matching cat image after $g$ generations.}
	\label{cat-6}
\end{figure}


\begin{table}[htbp]
\caption{The error after 1000 generations with different $k_{max}$ values}
\begin{center}
%\begin{booktabs}{colspec={ccccccc},row{even}={blue9}}
\begin{booktabs}{colspec={cccc}}
\toprule
{Image name} &
	{$k_{max}$} &
	Final error (\%) &
	Size (bytes)\\
\midrule
\SetCell[r=10]{c,1.5cm}{Sherlock} &   5 & 6.338 &  36\\
                                  &  10 & 4.218 &  70\\
                                  &  15 & 2.401 & 105\\
                                  &  20 & 1.439 & 139\\
                                  &  25 & 1.220 & 173\\
                                  &  30 & 1.301 & 207\\
                                  &  40 & 1.123 & 276\\
                                  &  50 & 1.006 & 345\\
                                  &  60 & 0.940 & 414\\
                                  &  70 & 0.898 & 483\\
\specialrule{2.5pt,gray5}{1pt}{1pt}
\SetCell[r=10]{c,1.5cm}{Cat}      &   5 & 3.775 &  36\\
                                  &  10 & 2.260 &  70\\
                                  &  15 & 1.320 & 105\\
                                  &  20 & 1.199 & 139\\
                                  &  25 & 0.837 & 173\\
                                  &  30 & 0.807 & 207\\
                                  &  40 & 0.532 & 276\\
                                  &  50 & 0.612 & 345\\
                                  &  60 & 0.569 & 414\\
                                  &  70 & 0.562 & 483\\
\bottomrule
\end{booktabs}
\label{tab1}
\end{center}
\end{table}


\section{Summary}

\bibliographystyle{IEEEtran}
\bibliography{IEEEabrv,references}

\end{document}
