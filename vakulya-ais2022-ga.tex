\documentclass[conference]{IEEEtran}
\IEEEoverridecommandlockouts
% The preceding line is only needed to identify funding in the first footnote. If that is unneeded, please comment it out.
\usepackage{cite}
\usepackage{amsmath,amssymb,amsfonts}
\usepackage{algorithmic}
\usepackage{graphicx}
\usepackage{textcomp}
\usepackage{xcolor}
%\usepackage{booktabs}
\usepackage{tabularray}
\UseTblrLibrary{booktabs}
\def\BibTeX{{\rm B\kern-.05em{\sc i\kern-.025em b}\kern-.08em
    T\kern-.1667em\lower.7ex\hbox{E}\kern-.125emX}}
\begin{document}

\title{Evaluation of a genetic algorithm based image approximation method}

\author{\IEEEauthorblockN{Gergely Vakulya}
\IEEEauthorblockA{\small \textit{Alba Regia Technical Faculty} \\
\textit{Óbuda University}\\
\textit{Székesfehérvár, Hungary}\\
\textit{vakulya.gergely@amk.uni-obuda.hu}
}}

\maketitle

\begin{abstract}
In this paper the design parameters of a
genetic algorithm (GA) based image
approximation / compression algorithm are
analyzed. The efficiency using different
approximation shapes and with different GA
parameters (population size, mutation rate)
are evaluated using test runs. Recommended
parameter sets for different scenarios are
provided.
\end{abstract}

\begin{IEEEkeywords}
genetic algorithm, image approximation
\end{IEEEkeywords}

\section{Introduction}

The image approximation algorithm presented in TODO used arbitrary triangles to
fill areas of the image. Using different shapes, however, can lead to different
results. In this paper rectangles and circles will be used for the approximation.

The required amount of data to represent the size and the position of the shapes
is different in each case. In case of triangles the coordinates of the three
vertices are required to store, which means a total of 6 numerical parameters.
A circle can be defined with the center point and the radius, summing up to 3 parameters.
Rectangles can be given with the two opposite vertices, and an optional value
for the rotation, resulting in 4 or 5 parameteres. Finally, the shapes can be mixed,
as well.

The following design parameters and settings will be evaluated.

\begin{itemize}

  \item{Fitness vs the number of generations using different shapes}

  \item{Fitness vs. }

\end{itemize}

\section{Related work}


\section{The GA-based image compression}

\subsection{Problem statement}


\subsection{Chromosome representation}


\subsection{Fitness function}


\subsection{Genetic operators}


\subsection{Program flow}


\subsection{Software architecture}


\section{Test results}

Figures \ref{sherlock-6} and \ref{cat-6} 

\begin{figure}[htbp]
	\centering
	\includegraphics[width=0.35\textwidth]{fig/sherlock6.png}
	\caption{The best matching Sherlock image after $g$ generations.}
	\label{sherlock-6}
\end{figure}

\begin{figure}[htbp]
	\centering
	\includegraphics[width=0.35\textwidth]{fig/cat6.png}
	\caption{The best matching cat image after $g$ generations.}
	\label{cat-6}
\end{figure}


\begin{table}[htbp]
\caption{The error after 1000 generations with different $k_{max}$ values}
\begin{center}
%\begin{booktabs}{colspec={ccccccc},row{even}={blue9}}
\begin{booktabs}{colspec={cccc}}
\toprule
{Image name} &
	{$k_{max}$} &
	Final error (\%) &
	Size (bytes)\\
\midrule
\SetCell[r=10]{c,1.5cm}{Sherlock} &   5 & 6.338 &  36\\
                                  &  10 & 4.218 &  70\\
                                  &  15 & 2.401 & 105\\
                                  &  20 & 1.439 & 139\\
                                  &  25 & 1.220 & 173\\
                                  &  30 & 1.301 & 207\\
                                  &  40 & 1.123 & 276\\
                                  &  50 & 1.006 & 345\\
                                  &  60 & 0.940 & 414\\
                                  &  70 & 0.898 & 483\\
\specialrule{2.5pt,gray5}{1pt}{1pt}
\SetCell[r=10]{c,1.5cm}{Cat}      &   5 & 3.775 &  36\\
                                  &  10 & 2.260 &  70\\
                                  &  15 & 1.320 & 105\\
                                  &  20 & 1.199 & 139\\
                                  &  25 & 0.837 & 173\\
                                  &  30 & 0.807 & 207\\
                                  &  40 & 0.532 & 276\\
                                  &  50 & 0.612 & 345\\
                                  &  60 & 0.569 & 414\\
                                  &  70 & 0.562 & 483\\
\bottomrule
\end{booktabs}
\label{tab1}
\end{center}
\end{table}


\section{Summary}

\bibliographystyle{IEEEtran}
\bibliography{IEEEabrv,references}

\end{document}
