\documentclass[conference]{IEEEtran}
\IEEEoverridecommandlockouts
% The preceding line is only needed to identify funding in the first footnote. If that is unneeded, please comment it out.
\usepackage{cite}
\usepackage{amsmath,amssymb,amsfonts}
\usepackage{algorithmic}
\usepackage{graphicx}
\usepackage{textcomp}
\usepackage{xcolor}
%\usepackage{booktabs}
\usepackage{tabularray}
\UseTblrLibrary{booktabs}
\def\BibTeX{{\rm B\kern-.05em{\sc i\kern-.025em b}\kern-.08em
    T\kern-.1667em\lower.7ex\hbox{E}\kern-.125emX}}
\begin{document}

\title{Evaluation of a genetic algorithm based image approximation method}

\author{\IEEEauthorblockN{Gergely Vakulya}
\IEEEauthorblockA{\small \textit{Alba Regia Technical Faculty} \\
\textit{Óbuda University}\\
\textit{Székesfehérvár, Hungary}\\
\textit{vakulya.gergely@amk.uni-obuda.hu}
}}

\maketitle

\begin{abstract}
In this paper
\end{abstract}

\begin{IEEEkeywords}
genetic algorithm, image approximation
\end{IEEEkeywords}

\section{Introduction}

Genetic algirithms (GA) \cite{ga-book} offer good approximations to many
problems, where no fast algorithm is known to calculate
the optimal solutions. It models the natural selection,
where each member of a population (called specimen)
represents one possible solution.

GA starts with an initial population, which is a set of
possible solutions (e.g. empty or randomly generated).
In each cycle GA produces new specimens by applying
\emph{mutation} and \emph{crossover} operations on the
existing ones. Then the population is evaluated by
the \emph{selection} operator based on the \emph{fitness}
function.

In this paper a GA-based lossy image compression method is
presented. A number of triangles are used to approximate
a monochromatic image; the position of the triangles are
optimized with GA. The chromosome representation, the
mutation nad crossover operators and different selection
strategies are evaluated. The algorithm is capable to give
recognizable approximation of images in the 100 byte range,
which means an extreme compression in exchange for the
lower quality output.

\section{Related work}

Genetic algorithms are suitable to problems, where
there no efficient solution is available to calculate
the optimal (or, in many cases, any) solution.
An advantage of GA is the iterative operation, making it possible
to stop the optimization process in any time and use the current
best solution.

Scheduling is an everyday problem in industy \cite{ga-scheduling}
and transportation planning \cite{ga-transport}, where genetic
algorithms can offer a good alternative to e.g. linear programming.

A similar area covers the Vehicle Routing Problems (VRP).
A practical genetic algorithm based solution can be found in \cite{ga-vrp}.
It is closely related to the traveling salesman problem, which
can also be approximated with GA \cite{ga-tsp}.

A good application of genetic algorithm is calculating
the synaptic weights of neural networks. MarIO \cite{mar-io}
is an AI player to the Super Mario game.

Genetic algorithms are also suitable to generate program code
(genetic programming) to determine the behavior of e.g.
autonomous robots. A good example is \cite{ga-robocode},
where a specialla tailored programming language (Robocode)
is used to program virtual battle robots.

In economics portfolio optimization \cite{ga-portfolio}
and automatic trading \cite{ga-trading}
can be implemented with genetic algorithms.

In the area of image processing GA can be
efficiently used in combination with neural networks for e.g.
image segmentation \cite{ga-imgseg} or image enhancement \cite{ga-imgenh}.
Image compression tends to use fast algorithms to give an
indentical (lossless compression) or an indistinguishable
(lossy compression) representation of the input image,
in singnificantly less space \cite{image-proc}.

\section{The GA-based image compression}

\subsection{Problem statement}

An image $IMG$ is given as square ($n \times n$) matrix, where each pixel is
either black or white:

\begin{equation}
	IMG = (a_{ij}) \in {0,1}^{n \times n}
\end{equation}

An approximation is given as a set of $n$ triangles. The corner points of
the triangles are placed in the target image, in integer positions,
and can have either black or white color. The $k^\text{th}$ triangle
$t_k$ can be given as a quad:

\begin{equation}
	t_k = (A, B, C, c), A, B, C \in {1 \dots n}^2, c \in {0, 1}
\end{equation}

Time $I(t_k)$ image of triangle $t_k$ is a set of points, which are geometrically
between the corners:

\begin{equation}
	\begin{split}
	I(t_k) &= {a_{ij}},\\
	\exists \alpha, \beta &\in [0, 1]:\\
	i &= \beta(\alpha A_1 + (1-\alpha)B_1)+(1-\beta)C_1\\
	j &= \beta(\alpha A_2 + (1-\alpha)B_2)+(1-\beta)C_2
	\end{split}
\end{equation}

\subsection{Chromosome representation}

The chromosome representation of an approximation $Q$ is
a set of triangles, represented by a triplet $(k, T, C)$, where

\begin{itemize}

	\item{$k \in Z$ is the number of triangles used in the approximation,}

	\item{$T \in Z^{k \times 3 \times 2}$ is the set of triangles,
	given with 3 coordinate pairs,}

	\item{$C \in {0,1}^k$ is the set of triangle colors.}

\end{itemize}

\subsection{Fitness function}

The image form $R$ of an approximation $(k, T, C)$ is defined as follows:

\begin{equation}
	R_{i,j}(k, T, C)=
	\begin{cases}
		C_v,& \exists v: (i, j) \in I(T_v),\\
		    & \nexists w: w>v, (i, j) \in I(T_v),\\
				&1 \leq v, w \leq k\\
		0,  & \text{otherwise}
	\end{cases}
\end{equation}

Instead of the fitness function, where the more suitable
specimen have larger fitness, an $E$ error function is used, representing
the proportion of the mismatching area between the target image and the triangle
based approximation, as a percentage.

\begin{equation}
	\begin{gathered}
		E = \frac{\sum_{i,j} \left( \frac{IMG_{i,j}-R_{i,j}}{mn} \right)}{n^2}\cdot 100\\
		1 \leq i,j \leq n
	\end{gathered}
\end{equation}

\subsection{Genetic operators}

The three genetic operators (mutation, crossover and selection)
are implemented as follows. Three simple mutation types was
built into the program.

\begin{itemize}

	\item{\emph{M1} removes a randomly selected triangle}

	\item{\emph{M2} adds a triangle with random coordinates}

	\item{\emph{M3} modifies the position of 1, 2 or 3 corneres of a triangle}

\end{itemize}

The crossover operator takes triangles from two selected triangle-set
and takes triangles from both, randomly.

During the selection the fitness function is calculated and a defined
amount of the population is eliminated and overwritten by
newly generated specimen (by crossover).

\subsection{Program flow}



\subsection{Software architecture}

The image approximation algorithm is implemented in C,
without using external libraries (besides glibc). This
approach makes the code easily portable, but on the other
hand it makes handling of different image formats more
difficult. Thus, only loading and saving of BMP files
with predefined resolution and color depth are supported.

The test program prints a short information line after
each generation, which contains the generation number,
the error of the best matching approximation and the
current number of polygons. The best matching image is
saved after each generation, to give more visual feedback.

With the actual settings each approximated image takes

\begin{itemize}

\item{8 bits to store the number of polygons,}

\item{$3 \times 2 \times 9 \times k$ bits to store the triangle
coordinates,}

\item{and $k$ bits to store the colors of the triangles,}

\end{itemize}

sum up to $1+\frac{55}{8}k$ bytes.

\section{Test results}

To test the performance of the proposed image approximation
method, two moderately complex test images was selected,
as shown in Fig. \ref{fig-orig}. Both images have a
resolution og $512 \times 512$ pixels.

\begin{figure}[htbp]
	\centering
	\includegraphics[width=0.35\textwidth]{fig/originals.png}
	\caption{The selected test images.
	(a) Silhouette of Sherlock Holmes,
	(b) Silhouette of a cat.}
	\label{fig-orig}
\end{figure}

The approximation algorithm was tested with the selected
images with the following settings. The size of the population
8000. In each generation each specimen is mutated with the
\emph{M1}, \emph{M2} and \emph{M3} operators with 10\%, 50\%
and 80\% probability, respectively. The 1800 worst specimen
was overwritten with the crossover of randomly selected ones.
Note that this is an elitist setup, which can lead to a fast
convergence, at a risk of a less accurate final result.

The polygon limit was set to $k_{max}=5$, 10, 15, 20, 30, 40, 50,
60, 70 and 80. Each different setup was run through 1000 generations.

Figures \ref{sherlock-6} and \ref{cat-6} show the image with
the best fitness after 25, 50, 100, 200, 300 and 500 generations
with $k_{max}=20$ polygons. As can be clearly seen, the
accuracy of the approximation quickly increases first; both
images are recognizable after 200 generations, but only slight
improvements can be observed afterwards.

\begin{figure}[htbp]
	\centering
	\includegraphics[width=0.35\textwidth]{fig/sherlock6.png}
	\caption{The best matching Sherlock image after $g$ generations.}
	\label{sherlock-6}
\end{figure}

\begin{figure}[htbp]
	\centering
	\includegraphics[width=0.35\textwidth]{fig/cat6.png}
	\caption{The best matching cat image after $g$ generations.}
	\label{cat-6}
\end{figure}

Figures \ref{curve-sherlock} and \ref{curve-cat} show the change
of best observed fitness after each generation for the Sherlock
and the cat images. Different colors mean different maximal
triangle numbers. With extremely low polygon count the approximation
remains very rough. Adding more than approx\. 30 triangles
does not make any quantitative improvement, some extra detail
was observed up to 60 triangles. A recognizable image approximated
with 20 triangles takes 139 bytes of data.

\begin{figure}[htbp]
	\centering
		\resizebox{.45\textwidth}{!}{\input{fig.tex}}
	\caption{Change of the fitness function during the approximation of the Sherlock image.
	The smallest error after each generation is shown for different polygon limits.} %20 polygons
	\label{curve-sherlock}
\end{figure}

\begin{figure}[htbp]
	\centering
		\resizebox{.45\textwidth}{!}{% GNUPLOT: LaTeX picture with Postscript
\begingroup
  \makeatletter
  \providecommand\color[2][]{%
    \GenericError{(gnuplot) \space\space\space\@spaces}{%
      Package color not loaded in conjunction with
      terminal option `colourtext'%
    }{See the gnuplot documentation for explanation.%
    }{Either use 'blacktext' in gnuplot or load the package
      color.sty in LaTeX.}%
    \renewcommand\color[2][]{}%
  }%
  \providecommand\includegraphics[2][]{%
    \GenericError{(gnuplot) \space\space\space\@spaces}{%
      Package graphicx or graphics not loaded%
    }{See the gnuplot documentation for explanation.%
    }{The gnuplot epslatex terminal needs graphicx.sty or graphics.sty.}%
    \renewcommand\includegraphics[2][]{}%
  }%
  \providecommand\rotatebox[2]{#2}%
  \@ifundefined{ifGPcolor}{%
    \newif\ifGPcolor
    \GPcolortrue
  }{}%
  \@ifundefined{ifGPblacktext}{%
    \newif\ifGPblacktext
    \GPblacktexttrue
  }{}%
  % define a \g@addto@macro without @ in the name:
  \let\gplgaddtomacro\g@addto@macro
  % define empty templates for all commands taking text:
  \gdef\gplbacktext{}%
  \gdef\gplfronttext{}%
  \makeatother
  \ifGPblacktext
    % no textcolor at all
    \def\colorrgb#1{}%
    \def\colorgray#1{}%
  \else
    % gray or color?
    \ifGPcolor
      \def\colorrgb#1{\color[rgb]{#1}}%
      \def\colorgray#1{\color[gray]{#1}}%
      \expandafter\def\csname LTw\endcsname{\color{white}}%
      \expandafter\def\csname LTb\endcsname{\color{black}}%
      \expandafter\def\csname LTa\endcsname{\color{black}}%
      \expandafter\def\csname LT0\endcsname{\color[rgb]{1,0,0}}%
      \expandafter\def\csname LT1\endcsname{\color[rgb]{0,1,0}}%
      \expandafter\def\csname LT2\endcsname{\color[rgb]{0,0,1}}%
      \expandafter\def\csname LT3\endcsname{\color[rgb]{1,0,1}}%
      \expandafter\def\csname LT4\endcsname{\color[rgb]{0,1,1}}%
      \expandafter\def\csname LT5\endcsname{\color[rgb]{1,1,0}}%
      \expandafter\def\csname LT6\endcsname{\color[rgb]{0,0,0}}%
      \expandafter\def\csname LT7\endcsname{\color[rgb]{1,0.3,0}}%
      \expandafter\def\csname LT8\endcsname{\color[rgb]{0.5,0.5,0.5}}%
    \else
      % gray
      \def\colorrgb#1{\color{black}}%
      \def\colorgray#1{\color[gray]{#1}}%
      \expandafter\def\csname LTw\endcsname{\color{white}}%
      \expandafter\def\csname LTb\endcsname{\color{black}}%
      \expandafter\def\csname LTa\endcsname{\color{black}}%
      \expandafter\def\csname LT0\endcsname{\color{black}}%
      \expandafter\def\csname LT1\endcsname{\color{black}}%
      \expandafter\def\csname LT2\endcsname{\color{black}}%
      \expandafter\def\csname LT3\endcsname{\color{black}}%
      \expandafter\def\csname LT4\endcsname{\color{black}}%
      \expandafter\def\csname LT5\endcsname{\color{black}}%
      \expandafter\def\csname LT6\endcsname{\color{black}}%
      \expandafter\def\csname LT7\endcsname{\color{black}}%
      \expandafter\def\csname LT8\endcsname{\color{black}}%
    \fi
  \fi
    \setlength{\unitlength}{0.0500bp}%
    \ifx\gptboxheight\undefined%
      \newlength{\gptboxheight}%
      \newlength{\gptboxwidth}%
      \newsavebox{\gptboxtext}%
    \fi%
    \setlength{\fboxrule}{0.5pt}%
    \setlength{\fboxsep}{1pt}%
    \definecolor{tbcol}{rgb}{1,1,1}%
\begin{picture}(6800.00,5100.00)%
    \gplgaddtomacro\gplbacktext{%
      \csname LTb\endcsname%%
      \put(392,426){\makebox(0,0)[r]{\strut{}$0$}}%
      \csname LTb\endcsname%%
      \put(392,919){\makebox(0,0)[r]{\strut{}$2$}}%
      \csname LTb\endcsname%%
      \put(392,1413){\makebox(0,0)[r]{\strut{}$4$}}%
      \csname LTb\endcsname%%
      \put(392,1906){\makebox(0,0)[r]{\strut{}$6$}}%
      \csname LTb\endcsname%%
      \put(392,2399){\makebox(0,0)[r]{\strut{}$8$}}%
      \csname LTb\endcsname%%
      \put(392,2893){\makebox(0,0)[r]{\strut{}$10$}}%
      \csname LTb\endcsname%%
      \put(392,3386){\makebox(0,0)[r]{\strut{}$12$}}%
      \csname LTb\endcsname%%
      \put(392,3879){\makebox(0,0)[r]{\strut{}$14$}}%
      \csname LTb\endcsname%%
      \put(392,4373){\makebox(0,0)[r]{\strut{}$16$}}%
      \csname LTb\endcsname%%
      \put(392,4866){\makebox(0,0)[r]{\strut{}$18$}}%
      \csname LTb\endcsname%%
      \put(504,213){\makebox(0,0){\strut{}$0$}}%
      \csname LTb\endcsname%%
      \put(1098,213){\makebox(0,0){\strut{}$100$}}%
      \csname LTb\endcsname%%
      \put(1692,213){\makebox(0,0){\strut{}$200$}}%
      \csname LTb\endcsname%%
      \put(2286,213){\makebox(0,0){\strut{}$300$}}%
      \csname LTb\endcsname%%
      \put(2880,213){\makebox(0,0){\strut{}$400$}}%
      \csname LTb\endcsname%%
      \put(3474,213){\makebox(0,0){\strut{}$500$}}%
      \csname LTb\endcsname%%
      \put(4067,213){\makebox(0,0){\strut{}$600$}}%
      \csname LTb\endcsname%%
      \put(4661,213){\makebox(0,0){\strut{}$700$}}%
      \csname LTb\endcsname%%
      \put(5255,213){\makebox(0,0){\strut{}$800$}}%
      \csname LTb\endcsname%%
      \put(5849,213){\makebox(0,0){\strut{}$900$}}%
      \csname LTb\endcsname%%
      \put(6443,213){\makebox(0,0){\strut{}$1000$}}%
    }%
    \gplgaddtomacro\gplfronttext{%
      \csname LTb\endcsname%%
      \put(5574,4674){\makebox(0,0)[r]{\strut{}5}}%
      \csname LTb\endcsname%%
      \put(5574,4461){\makebox(0,0)[r]{\strut{}10}}%
      \csname LTb\endcsname%%
      \put(5574,4248){\makebox(0,0)[r]{\strut{}15}}%
      \csname LTb\endcsname%%
      \put(5574,4035){\makebox(0,0)[r]{\strut{}25}}%
      \csname LTb\endcsname%%
      \put(5574,3822){\makebox(0,0)[r]{\strut{}30}}%
    }%
    \gplbacktext
    \put(0,0){\includegraphics[width={340.00bp},height={255.00bp}]{fig_cat}}%
    \gplfronttext
  \end{picture}%
\endgroup
}
	\caption{Change of the fitness function during the approximation of the cat image.
	The smallest error after each generation is shown for different polygon limits.}
	\label{curve-cat}
\end{figure}

\begin{table}[htbp]
\caption{The error after 1000 generations with different $k_{max}$ values}
\begin{center}
%\begin{booktabs}{colspec={ccccccc},row{even}={blue9}}
\begin{booktabs}{colspec={ccc}}
\toprule
{Image name} &
	{$k_{max}$} &
	Final error (\%) \\
\midrule
\SetCell[r=10]{c,1.5cm}{Sherlock} &   5 & 6.338 \\
                                  &  10 & 4.218 \\
                                  &  15 & 2.401 \\
                                  &  20 & 1.439 \\
                                  &  25 & 1.220 \\
                                  &  30 & 1.301 \\
                                  &  40 & 1.123 \\
                                  &  50 & 1.006 \\
                                  &  60 & 0.940 \\
                                  &  70 & 0.898 \\
\specialrule{2.5pt,gray5}{1pt}{1pt}
\SetCell[r=10]{c,1.5cm}{Cat}      &   5 & 3.775 \\
                                  &  10 & 2.260 \\
                                  &  15 & 1.320 \\
                                  &  20 & 1.199 \\
                                  &  25 & 0.837 \\
                                  &  30 & 0.807 \\
                                  &  40 & 0.532 \\
                                  &  50 & 0.612 \\
                                  &  60 & 0.569 \\
                                  &  70 & 0.562 \\
\bottomrule
\end{booktabs}
\label{tab1}
\end{center}
\end{table}

\section{Summary}

A genetic algorithm based image approximation method
was proposed. The algorithm can approximate black and
white images with a limited (definable) number of
bkack and white triangles. The GA is implemented
with 3 mutation and a crossover operator. Selection
operator uses an elitist approach. Formal definition
was given to the fitness function computation.

The proposed algorithm was implemented as a C
language program and tested with different input
images and with several maximal number number
of triangles.

The algorithm can generate a recognizable quality
approximation of an image with moderate complexity
in less than 150 bytes with computation times
under 10 miutes.

\begin{thebibliography}{00}

\bibitem{ga-book}Sivanandam, S. N. and Deepa, S. N., ``Introduction to Genetic Algorithms'',
	Springer-Verlag Berlin Heidelberg, 2008, ISBN 978-3-540-73189-4

\bibitem{mar-io}Baldominos A., Saez Y., Recio G., Calle J. ``Learning Levels of Mario AI Using Genetic Algorithms``,
	In: Advances in Artificial Intelligence. CAEPIA 2015. Lecture Notes in Computer Science, vol 9422. Springer, Cham. doi: 10.1007/978-3-319-24598-0\_24

\bibitem{ga-transport}C. Lin, J. Yu, J. Liu and C. Lee, ``Genetic Algorithm for Shortest Driving Time in Intelligent Transportation Systems``, 2008 International Conference on Multimedia and Ubiquitous Engineering, 2008, pp. 402-406, doi: 10.1109/MUE.2008.16.

\bibitem{ga-vrp}Zhu, K. Q., ``A new genetic algorithm for VRPTW``, Proceedings of the international conference on artificial intelligence. 2000.

\bibitem{ga-tsp}Ahmed, Z. H. ``Genetic algorithm for the traveling salesman problem using sequential constructive crossover operator``, International Journal of Biometrics \& Bioinformatics (IJBB) 3.6 (2010): 96.

\bibitem{ga-scheduling}Lim, C., and Eoksu S., ``Production planning in manufacturing/remanufacturing environment using genetic algorithm``, Proceedings of the 7th annual conference on Genetic and evolutionary computation. 2005.

\bibitem{ga-robocode}Shichel Y., Ziserman E., Sipper M., ``Using Genetic Programming to Evolve Robocode Players`` In: Genetic Programming. EuroGP 2005. Lecture Notes in Computer Science, vol 3447. Springer, Berlin, Heidelberg. https://doi.org/10.1007/978-3-540-31989-4\_13

\bibitem{ga-portfolio}S. Slimane and M. Benbouziane, ``Portfolio Selection Using Genetic Algorithm`` In: Journal of Applied Finance \& Banking , Vol. 2, No. 4 (2012): pp. 143-154.

\bibitem{ga-trading}R.J. Kuo, C.H. Chen, Y.C. Hwang, ``An intelligent stock trading decision support system through integration of genetic algorithm based fuzzy neural network and artificial neural network``,
Fuzzy Sets and Systems,
Volume 118, Issue 1,
2001,
Pages 21-45,
ISSN 0165-0114,
10.1016/S0165-0114(98)00399-6.

\bibitem{ga-imgseg}Pare, S., Bhandari, A. K., Kumar, A., Singh, G. K., Khare, S. ``Satellite image segmentation based on different objective functions using genetic algorithm: a comparative study.`` In 2015 IEEE international conference on digital signal processing (DSP) (pp. 730-734). IEEE.

\bibitem{image-proc}Jayaraman, S., S. Esakkirajan, and T. Veerakumar. ``Digital image processing``, Mc Graw Hill India, 2009, ISBN 978-0070144798

\bibitem{ga-imgenh}Deborah, H., Arymurthy, A. M., ``Image enhancement and image restoration for old document image using genetic algorithm``, In 2010 Second International Conference on Advances in Computing, Control, and Telecommunication Technologies (pp. 108-112). IEEE.

\end{thebibliography}

\end{document}
