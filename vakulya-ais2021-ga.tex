\documentclass[conference]{IEEEtran}
\IEEEoverridecommandlockouts
% The preceding line is only needed to identify funding in the first footnote. If that is unneeded, please comment it out.
\usepackage{cite}
\usepackage{amsmath,amssymb,amsfonts}
\usepackage{algorithmic}
\usepackage{graphicx}
\usepackage{textcomp}
\usepackage{xcolor}
%\usepackage{booktabs}
\usepackage{tabularray}
\UseTblrLibrary{booktabs}
\def\BibTeX{{\rm B\kern-.05em{\sc i\kern-.025em b}\kern-.08em
    T\kern-.1667em\lower.7ex\hbox{E}\kern-.125emX}}
\begin{document}

\title{Evaluation of a genetic algorithm based image approximation method}

\author{\IEEEauthorblockN{Gergely Vakulya}
\IEEEauthorblockA{\small \textit{Alba Regia Technical Faculty} \\
\textit{Óbuda University}\\
\textit{Székesfehérvár, Hungary}\\
\textit{vakulya.gergely@amk.uni-obuda.hu}
}}

\maketitle

\begin{abstract}
In this paper
\end{abstract}

\begin{IEEEkeywords}
genetic algorithm, image approximation
\end{IEEEkeywords}

\section{Introduction}

Genetic algirithms (GA) \cite{b1} offer good approximations to many
problems, where no fast algorithm is known to calculate
the optimal solutions. It models the natural selection,
where each member of a population (called specimen)
represents one possible solution.

GA starts with an initial population, which is a set of
possible solutions (e.g. empty or randomly generated).
In each cycle GA produces new specimens by applying
\emph{mutation} and \emph{crossover} operations on the
existing ones. Then the population is evaluated by
the \emph{selection} operator based on the \emph{fitness}
function.

In this paper a GA-based lossy image compression method is
presented. A number of triangles are used to approximate
a monochromatic image; the position of the triangles are
optimized with GA. The chromosome representation, the
mutation nad crossover operators and different selection
strategies are evaluated. The algorithm is capable to give
recognizable approximation of images in the 150 byte range.

\section{Implementation}

\subsection{Software architecture}

The image approximation algorithm is implemented in C,
without using external libraries (besides glibc). This
approach makes the code easily portable, but on the other
hand it makes handling of different image formats more
difficult. Thus, only loading and saving of BMP files
with predefined resolution and color depth are supported.

\subsection{Chromosome representation}

\subsection{Genetic operators}


\section{Test results}

\subsection{Test images}

\begin{figure}[htbp]
	\centering
	\includegraphics[width=0.35\textwidth]{fig/sherlock-orig.png}
	\caption{The selected test image}
	\label{sherlock-orig}
\end{figure}

\begin{figure}[htbp]
	\centering
		\resizebox{.45\textwidth}{!}{\input{fig.tex}}
	\caption{TODO}
	\label{foo}
\end{figure}


\section{Summary}


\begin{thebibliography}{00}

\bibitem{b1}Sivanandam, S. N. and Deepa, S. N., ``Introduction to Genetic Algorithms'',
	Springer-Verlag Berlin Heidelberg, 2008, ISBN 978-3-540-73189-4
	%
%\bibitem{b2}ISO 516:2019. Camera shutters — Timing — General definition and
%	mechanical shutter measurements. International Organization for Standardization,
%	2019. Online: https://www.iso.org/obp/ui/\#iso:std:iso:516:ed-4:v1:en
%
%\bibitem{b3}Y. Asakura, et al., ``Exposure precision tester and exposure precision
%	testing method for camera'', U.S. Patent 5 895 132, Apr. 20, 1999.
%
%\bibitem{b4}Peter D. Hiscocks, ``Measuring Camera Shutter Speed'',
%	Royal Astronomical Society, Toronto Centre, Canada, 2010.
%	Online: https://www.ee.ryerson.ca/~phiscock/astronomy/light-pollution/shutter-cal.pdf
%
%\bibitem{b5}V. N. Budilov, V. I. Volovach, M. V. Shakurskiy and S. V. Eliseeva,
%``Automated measurement of digital video cameras exposure time'',
%	East-West Design \& Test Symposium (EWDTS 2013), Rostov-on-Don,
%	2013, pp. 1-4. doi: 10.1109/EWDTS.2013.6673136
%
%
%\bibitem{b6}CCTVCAD Lab Toolkit. CCTVCAD Software. Perm, Russia, 2011.
%Online: http://www.cctvcad.com/labtoolkit\_help
%
%\bibitem{b7}L. Masson, F. Cao, C. Viard, and F. Guichard,
%	``Device and algorithms for camera timing evaluation''
%	in Proc. IS\&T/SPIE Electronic Imaging Symposium,
%	San Francisco, California, United States, 2014.
%	doi:10.1117/12.2042161
%
%\bibitem{b8}Image Engineering: LED-Panel.
%	Online: https://www.image-engineering.de/products/equipment/measurement-devices/900-led-panel
%
%\bibitem{b9}G. Shize, S. Shenghe and Z. Zhongting,
%	``A novel equivalent sampling method using in the digital storage oscilloscopes''
%	in Proc. 1994 IEEE Instrumentation and Measurement Technology Conference,
%	Hamamatsu, Japan, 1994, pp. 530-532 vol.2.
%	doi: 10.1109/IMTC.1994.351901
%
%\bibitem{b10}A. G. J. Holt, J. J. Hill and R. Linggard,
%	``Integral sampling'',
%	in Proceedings of the IEEE, vol. 61, no. 5, pp. 679-680, May 1973.
%	doi: 10.1109/PROC.1973.9138
%
%\bibitem{b11}FLIR Machine Vision Cameras.
%	Online: https://www.flir.com/browse/industrial/machine-vision-cameras/

\end{thebibliography}

\end{document}
